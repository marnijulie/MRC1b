\documentclass[11pt]{article}

\usepackage{amsmath}
\usepackage{amsfonts}
\usepackage{amssymb}
\usepackage{amsthm,epic}
\usepackage{mathrsfs}

\usepackage{tikz} 
\usepackage[T1]{fontenc}
\usepackage{enumitem}
\usepackage{graphicx, xcolor}
\usepackage{footnote,pifont}
\usetikzlibrary{patterns}
\renewcommand{\labelenumi}{\theenumi}
\renewcommand{\theenumi}{{\rm(\roman{enumi})}}
\renewcommand{\geq}{\geqslant}
\renewcommand{\leq}{\leqslant}
\newtheorem{thm}{Theorem}
\newtheorem{defn}[thm]{Definition}
\newtheorem{rem}[thm]{Remark}
\newtheorem{cor}[thm]{Corollary}
\newtheorem{proposition}[thm]{Proposition}
\newtheorem{lem}[thm]{Lemma}
\newtheorem{que}{Problem}
\newtheorem{conjecture}[que]{Conjecture}

%------------------------------------------------------------------
% Document formatting

\usepackage[T1]{fontenc}
\usepackage[mathscr]{eucal}
\usepackage{fullpage}

\newcommand\mjm[1]{\mbox{}
{\marginpar{\color{red!50}$\EuScript{M}$}}
{\it\noindent\color{red!50}#1}}%


%------------------------------------------------------------------
% Math Notation
\newcommand{\cT}{\EuScript{T}}


\author{MRC1b Group\footnote{using computations of S Melczer and J Courtiel}}
%------------------------------------------------------------------
\title{{\sc The Tandem Model}\\{\sl \color{gray}Working Notes}}
%------------------------------------------------------------------
\begin{document}
\maketitle

\section{Introduction}
Consider the model of lattice path walks defined by steps from the set
$\cT=\{ [1,0], [-1,1], [0,-1] \}$ restricted to the first quadrant. In
the literature, these have taken the name the ``Tandem Model'' owing
to the fact that they can model two queues operating in tandem. 
The generating function of this model is algebraic, and there is an
explicit form. 

Our goal here is two-fold: First, we would are interested in the
enumeration of walks in {\it weighted\/} tandem models in a form such
that generic weights appear in the formula. This in turn permits a
clear picture of the reliance of the final asymptotic formulas on the
drift of the initial model. Secondly, we strive to understand the
interaction and connections between several different techniques. Can
the use of multiple techniques simultaneously facilitate the
computations? This may imply easier computations, nicer forms for
answers, or suggest means for random generation. In each case we
derive expressions for the dominant singularity of the generating
function, and we compare them. It also allows us ot interpolate
between $Q(0,0)$ and $Q(1,0)$.

\mjm{Once we have a conclusion, we can add to this.}
\begin{figure}\center
{\Huge X}
\caption{A random tandem walk}
\end{figure}
\section{The unweighted case}
\subsection{Background}
\mjm{Connection to Young tableau of bounded height}

There is a simple size preserving bijection from the set of walks in
the tandem model to standard Young tableaux of height bounded by 3.
Recall, a standard Young tableau is a filling of a Ferrer's diagram
that is strictly increasing along the rows and columns.

We denote the steps of a tandem walk by $N$, $SE$, $W$ and remark that
the condition of staying in the quarter plane implies that at any
point in the walk the number of $W$ is at most the number of $SE$
steps, which is in turn bounded by the number of $N$ steps.  The
aforementioned bijection is as follows. The image of a walk
$w=(w_1, w_2, \dots, w_n)\in \{N, SE, W\}^n$ is a filling of the
partition $(\lambda_1, \lambda_2, \lambda_3)$ of $n$, where
$\lambda_1$ is the number of $N$ steps, $\lambda_2$ is the number of
$SE$ steps and $\lambda_3$ is the number of $W$ steps. The quarter
plane condition ensures that this is a Ferrers diagram. To fill this
diagram, parse the $w$. If $w_i=N$, (resp. $SE$, $W$) then the number
$i$ is added to the top row (resp. second, third).  Remark that the
endpoint of a walk is $(\#N-\#SE, \#SE -\#W)$.

For example the image of $w=(N, N, SE, N, W, SE, N)$, which is a
tandem walk that ends at $(2, 1)$ is the tableau
$(1,2,4,7)(3,6)(5)$. \mjm{Format this.}

In fact, there are many combinatorial classes in bijection. Many of
them are listed in the OEIS entry A001006. For example, number of ways
of drawing any number of nonintersecting chords joining $n$ (labelled)
points on a circle is the same as the number of tandem walks of length~$n$. 

\subsection{Exact enumeration}
Let~$a_{i,j}(n)$ denote the number of walks with steps from $\cT$,
starting at the origin, ending at the point $(i,j)$ and staying in the
first quadrant. We can use the bijection to derive an explicit formula
for~$a_{i,j}(n)$. 

The enumeration of standard Young tableaux is a well explored topic,
and in this case, we use the well-known hook formula to enumerate
tableaux of a given shape. The number of standard young tableaux of
shape $\lambda$ is given by
\[|SYT(\lambda)| = \frac{n!}{\prod_{x\in[\lambda]} h_x}\]
where the product is taken over all boxes $x$ in the ferrer's diagram,
and $h_x$ is the sum of the boxes to the right and below the box $x$
in the diagram.

If a walk ends at point $(i,j)$ and is of length $n$, then $n=\#N
+\#SE+\#W$ and $i=\#N-\#SE$ $j=\#SE -\#W$ so it is in bijection 
with a partition of shape $\lambda=(n+2i+j, n-i+j, n-i-2j)$.

\mjm{I don't recall how to translate the hook length formula into the one below.
Remark: $n-i-2j=3 \#W$, $n+2i+j=3\#N$, $n-i+j=3\#S$, hence it is 
of the shape $\lambda=(n+2i+j, n-i+j, n-i-2j)$}

\begin{equation}
\label{eqn:exact}a_{ij}(n)=
\begin{cases}{\frac { \left( i+1 \right)  \left( j+1 \right)  \left( i+j+2 \right) 
n!}{ \left( n/3-i/3-2/3\,j \right) !\, \left( n/3-i/3+j/3+1 \right) !
\, \left( n/3+2/3\,i+j/3+2 \right) !}}
& \mbox{if } 3|n-i-2j, 3|n-i+j, \mbox{ and } 3|n+2i+j \\
0 & \mbox{otherwise}.
\end{cases}
\end{equation}

\subsection{Generating function expression}
\mjm{Expand upon this}
\[
Q(1,1; t) = \frac{ 1 - t - \sqrt{1-2t-3t^2}}{2t^2}.
\]
Exponential GF: $\exp(t)\operatorname{BesselI}(1, 2t)/t$.

\subsection{Functional equation}
Although we consider several strategies, each relies on an analysis of the
\emph{complete generating function\/} defined as the formal series in
$\mathbb{Q}[x, y][\,[t]\,]$: 
\[
Q(x,y;t)=\sum_{i,j,n} a_{ij}(n)\, x^i y^j t^n.
\]
We remark that several evaluations are of interest. The series
$Q(x,0;t)$ marks the walks that return the $x$-axis and tracks the
position.  The series $Q(0,0;t)$ is the generating function of
\emph{excursions}, that is walks that return to $(0,0)$, and the
series $Q(1,1;t)$ is the generating function of walks that end
anywhere.

The complete generating function satisfies the following functional
equation  
\begin{equation}
\label{eqn:kernel}
K(x,y)Q(x,y;t)=xy - K(x,0)Q(x,0;t) -K(0,y)Q(0,y;t) \quad \mbox{with}
\quad K(x,y)= xy(1-t(x+\frac{y}{x} +\frac{1}{y})).
\end{equation}

\subsection{Differential equation}
Given a series development, it is possible to guess a candidate
differential equation for which the series is a solution. This is
implemented in Maple using the {\tt gfun} package~\cite{}, for example. The
initial series development of $Q(x,y;t)$ is straightfoward to compute
using a basic recurrence. This process is robust enough to handle
additional variables, although practically speaking, we were only able
to guess differential equations for functions with one additional
variable. Using the differential equations guessed for $K(x,0)Q(x,0)$
and $K(0,y)Q(0,y)$, using equation Eqn.~\eqref{eqn:kernel}, and the
closure properties of differential equations, we found a candidate
differential equation satisfied by $K(x,y)Q(x,y;t)$. These are all
located in the appendix. \mjm{Put these in the appendix. PRobably not
  the one for $Q(x,y;t)$ though.}

The dominant singularity of the generating function is a root of the
leading term. Once the value is identified, a local solution expansion
around this point gives the sub-exponential factor in the asymptotic
growth.

The leading coefficient \mjm{mention the order} for the candidate DE for $K(x,0)Q(x,0)$ is:
\begin{align*}
p(x,t)&={t}^{4} \left( 3\,t-1 \right)  \left( 9\,{t}^{2}+3\,t+1 \right) 
 \left( {x}^{3}{t}^{2}-2\,t{x}^{2}-4\,{t}^{2}+x \right) \\& \qquad \qquad \left( 63\,{t
}^{4}{x}^{3}-5\,{x}^{4}{t}^{2}-99\,{t}^{3}{x}^{2}+288\,{t}^{4}+5\,{x}^
{3}t-52\,x{t}^{2}+4\,{x}^{2} \right)
\\&= {t}^{4} \left( 3\,t-1 \right) \, p_1(x,t)\,  p_2(x,t)\,  p_3(x,t).
\end{align*}
We are interested in positive real valued solutions to $p(x,t)$ for fixed
non-negative real values of $x$. Such a value exists by Pringsheim's
theorem. These are the candidates for dominant singularity. 
When $x=0$ the only real postive root is $t=1/3$. Thus, the dominant
singularity of $Q(0,0;t)$ is $1/3$. There are two real roots when
$x=1$, $t=1/3$ and 
\[t={\frac { \left( 25381+819\,\sqrt {518} \right) ^{2/3}-2\,\sqrt [3]{
25381+819\,\sqrt {518}}+667}{117\,\sqrt [3]{25381+819\,\sqrt {518}}}}
\sim 0.4461\dots.\] The latter is a root of $p_3(1,t)$. 

\mjm{Is it fair to say that from this process we cannot decide which
  one is the dominant singularity?}

Other evaluations will be important for the weighted case, as we shall
see in Section~\ref{sec:WeightedDE}. Most important, in fact, will be
how the dominant singularity changes as a function of $x$. The point
$x=1$ is very important, and behaviour changes both in this set up,
and in the solution given by an integral.

\mjm{
\begin{enumerate}
\item Note how many terms would have been necessary for the $Q(x,y;t)$
  DE. Can we do this compuation?
\item Does desingularization help?
\item We need to prove that the DEs are correct. Possibilities: (1)
  Using the explicit form of $a_{ij}(n)$ of Equation~\ref{eqn:exact},
  possibly with computer algebra, if the computations get too messy;
  (2) Try to get a diagonal form for $Q(x,0)$ (3) Are there bounds
  that we can compute?.
\end{enumerate}

}



\subsection{Analytic expression for $Q(x,0)$ as an integral}
Since the tandem model is a small step model, it fits into the
framework of Fayolle/Kurkova/Raschel.  As a first result, since the
drift is zero, by Fayolle/Raschel~\cite{FaRa10} the dominant
singularity is the number of steps, that is $\rho=\frac{1}{3}$.

We can also describe the integral form. 


The following expressions for $Q(x,0)$ and $Q(0,y)$ can be found in \cite[Theorem 1]{Ra12} (with $(i-1,j-1)$ denoting the starting point of the walk):
\begin{equation}
\label{eq:expression_K(x,0)Q(x,0)_Ra12}
     \left\{\begin{array}{lll}
     K(x,0)Q(x,0) &=&\displaystyle \frac{1}{2\pi i}\int_{\mathcal M} u^iY_0(u)^j\frac{w'(u)}{w(x)-w(u)}\text{d}u,\medskip\\
     K(0,y)Q(0,y) &=&\displaystyle \frac{1}{2\pi i}\int_{\mathcal L} X_0(u)^{i}u^j\frac{\widetilde w'(u)}{\widetilde w(y)-\widetilde w(u)}\text{d}u,
     \end{array}\right.
\end{equation}
provided that $w(0)=\widetilde w(0)=\infty$, which will hold with our
choice \eqref{eq:function_w_w_tilde}.

\[Y_1(x)=1/2\,{\frac {-t+x+\sqrt {-4\,{t}^{2}{x}^{3}+{t}^{2}-2\,tx+{x}^{2}}}{t
x}}\quad Y_0(x)=-1/2\,{\frac {t-x+\sqrt {-4\,{t}^{2}{x}^{3}+{t}^{2}-2\,tx+{x}^{2}}
}{tx}}\]
\[X_1(y)=1/2\,{\frac {-t{y}^{2}+y+\sqrt {{t}^{2}{y}^{4}-2\,t{y}^{3}-4\,{t}^{2}
y+{y}^{2}}}{t}}\quad X_0(y)=
-1/2\,{\frac {t{y}^{2}+\sqrt {{t}^{2}{y}^{4}-2\,t{y}^{
3}-4\,{t}^{2}y+{y}^{2}}-y}{t}}\]



 $\mathcal M$ is the curve $X([y_1,y_2])$ and $\mathcal L=Y([x_1,x_2])$. Equation \eqref{eq:expression_K(x,0)Q(x,0)_Ra12} provides contour integral expressions. In particular, with this formula the generating functions are defined only in the interior of the curves $\mathcal M$ and $\mathcal L$.
The conformal gluing functions $w$ and $\widetilde w$ are defined by
(see \cite[Theorem 3 (iv)]{Ra12})

\begin{equation}
\label{eq:function_w_w_tilde}
     \left\{\begin{array}{lll}
     w(x) &=& \dfrac{t}{x^2}-\dfrac{1}{x}-tx,\medskip\\
     \widetilde w(y) &=& ty^2-y-\dfrac{t}{y}.
     \end{array}\right.
\end{equation}
Functions $w$ and $\widetilde w$ satisfy $w=\widetilde w(Y_0)$ and $\widetilde w=w(X_0)$, and in particular $w(X_0(Y_0))=w$.

\mjm{Update the caption for the figure}
\begin{figure}\center
\includegraphics[width=.3\textwidth]{curveX-singular.jpg}$\quad$
\includegraphics[width=.3\textwidth]{curveX.jpg}
\caption{The curve $\mathcal M$ for $t=1/3$ and $t=1/3-\epsilon$}
\end{figure}

\subsubsection{Singularities of $Q(0,0)$}


\subsubsection{Analytic continuation}

One has
\begin{equation}
\label{eq:eq(4.4)}
     K(x,0)Q(x,0)=K(X_0(Y_0(x)),0)Q(X_0(Y_0(x)),0)+Y_0(x)(x-X_0(Y_0(x))).
\end{equation}
See \cite[Equation (4.4)]{FaRa-12} for the original derivation of the above identity.

The computation of $Q(\alpha,0;\gamma t)$ and $Q(0,\beta;\gamma t)$
depend on the location of $\alpha$ and $\beta$ w.r.t.\ the curves
$\mathcal M$ and $\mathcal L$. If there are inside we can use the
integral expressions given in
\eqref{eq:expression_K(x,0)Q(x,0)_Ra12}. If not, one has to use first
\eqref{eq:eq(4.4)}.

\subsection{Orbit Sum}
This is a candidate for the orbit sum method, and it was computed by
Bousquet-M\'elou and Mishna~\cite{BoMi10}.

\paragraph{Group of the walk}
The group is generated by two involutions, and is of order 6.
\[[[[\iota,\psi,\iota],[{y}^{-1},{x}^{-1}]],[[\psi,\iota],[{\frac {y}{x}
},{x}^{-1}]],[[\iota,\psi],[{y}^{-1},{\frac {x}{y}}]],[[\psi],[x,{
\frac {x}{y}}]],[[\iota],[{\frac {y}{x}},y]],[[],[x,y]]]\]
\paragraph{Positive series extraction expression}
\begin{equation}\label{eqn:OrbitSum}
Q(1,1;t)=[x^{>0}][y^{>0}] {\frac {-{y}^{3}{x}^{3}+{x}^{4}y+x{y}^{4}-{x}^{3}-{y}^{3}+xy}{{x}^{2}{
y}^{2} \left( t{x}^{2}y+t{y}^{2}+tx-xy \right) }}
\end{equation}


\begin{que}
We can express the generating function for $Q(x,0;t)$ as a diagonal. Use this
to verify the differential equation. The program of mathematica. 
\end{que}

\subsection{ACSV}
\mjm{Put in the details from the expression for the counting generating function 
\cite{MeWiXX}
}


\subsection{Creative Telescoping}
Has solution as an integral of hypergeometric functions (van Heoij)
\[Q(1,0;t)=(1+t)^{1/2}(1-3t)^{3/2}Int((hypergeom([-2/3,-1/3],[1],27t)-1+4t(hypergeom([-1/3,1/3],[2],27t^3)-1)+9t^2)/(3(1+t)^{3/2}(1-3t)^{5/2}),t)/t^3.\]

\mjm{Get this entry from the table of Chyzak.}


\section{The weighted case}
The weighted version of a model uses the same directions and assigns
non-negative real valued weights. When the weights are
integers, it can be interpreted as multiple distinct copies of the
same step. When they are positive and sum to one, it is a probability.
Kauers and Yatchuk~\cite{KaYa15} determined that all possible
non-negative weightings of this model will lead to D-finite generating
functions.

Let the weights, respectively, be $w=(\alpha, \beta, \gamma)$ for
$[1,0]$ (E), $[-1,1]$ (NW) and $[0,-1]$ (S).  The weight of a walk is
the product of the weight of each step. For example the weight of the
walk E E NW S is $\alpha^2\beta\gamma$. We are interested in the series
\begin{equation}
\label{eq:Qw}
Q^{(w)}(x,y;t)=\sum_{w\in \mathcal{W}} \mbox{weight}(w)\, x^{x(w)}y^{y(w)} t^{\mbox{length}(w)}.
\end{equation}

\subsection{From weights to parameters}
The generating function for the weighted model can be expressed by an
algebraic substitution evaluation of the unweighted model. 

\begin{proposition} The complete generating function for $\cT^{(w)}$
  the weighted tandem model with weight vector $(\alpha, \beta,
  \gamma)$, denoted $Q^{(w)}(x,y;t)$ satisfies:
\begin{equation}
Q^{(w)}(x,y;t)=
Q\left(\frac{\alpha}{\gamma\beta}x,\frac{\beta}{\gamma}y;\frac{t}{\alpha\beta\gamma}\right).
\end{equation}
Here $Q(x,y;t)$ is the generating function for the unweighted model,
equivalently for $w=(1,1,1)$. 
\end{proposition}
\begin{proof}
\mjm{Proof using the explicit formula.}
\end{proof}

We recall Lemma~16 from~\cite{AlBo14}. If $F(a,u)=\sum f_n(a) u^n$ is
a formal power series in $u$ with coefficients in $\mathbb{N}[a]$ such
that $f_n(a)$ has degree at most $n$. Assume further that~$F$ is not a
polynomial.
 
For $a\geq 0$, let $\rho(a)$ be the radius of convergence of the
series $F(a, \cdot)$. Then $\rho$ is a non-increasing function on $[0,
\infty)$ which is finite and and continuous on $(0, +\infty)$. 

This means that the radius of convergence of $Q(x,0;t)$ as a series in
$t$ is continuous as a function of $x$.

\subsection{Analysis of differential equations}
\label{sec:WeightedDE}
As we noted earlier, the radius of convergence of $Q(x,0;t)$ as a
function of $t$ can be expressed as a function of $x$ for non-negative
real values of $x$.

\mjm{I think a few well chosen figures would make this very clear.}

\begin{figure}\center
\includegraphics[width=.45\textwidth]{TDomSing.png}
\includegraphics[width=.45\textwidth]{TDomSing2.png}
\caption{{\it Left. } Solutions of $p_3(x,t)$ for $0\leq x$ and
  $t$. {\it Right. }Real solutions to factors of $p(x,t)$}
\label{fig:TDomSing}
\end{figure}


\subsection{Integral form}
The integral form in Equation~\eqref{eq:function_w_w_tilde} holds only
for $|x|<1$. This means that we can only find an expression for
certain weightings using this strategy.
\begin{figure}\center
\includegraphics[width=.45\textwidth]{X0-roots.jpg} $\qquad$
\includegraphics[width=.45\textwidth]{Y0-roots.jpg}
\caption{{\it Left. } Singularities coming from $X_0(x)$ for different
  values of $x$. 
  $t$. {\it Right. }Singularities coming from $Y_0(y)$ for different
  values of $y$. }
\label{fig:TDomSing}
\end{figure}


\subsection{Orbit Sum}
The orbit sum still holds under the weightings, and we can find an
expression for the generating function as a diagonal using the
parameterized group


\paragraph{The group}
\begin{equation}
[[[\iota,\psi,\iota],[{\frac {\gamma}{y\alpha}},{\frac {\gamma}{\alpha
\,x}}]],[[\psi,\iota],[{\frac {\beta\,y}{\alpha\,x}},{\frac {\gamma}{
\alpha\,x}}]],[[\iota,\psi],[{\frac {\gamma}{y\alpha}},{\frac {\gamma
\,x}{\beta\,y}}]],[[\psi],[x,{\frac {\gamma\,x}{\beta\,y}}]],[[\iota],
[{\frac {\beta\,y}{\alpha\,x}},y]],[[],[x,y]]]
\end{equation}

\paragraph{The orbit sum}
\begin{equation}
\label{eq:WtOrbitSum}
Q^{(w)}(x,y;t)=[x^{\geq 0}][y^{\geq 0}]-{\frac {1}{\alpha\,t{x}^{2}y+\beta\,t{y}^{2}+\gamma\,tx-xy} \left( -{
\frac {{\gamma}^{2}}{y{\alpha}^{2}x}}+{\frac {\beta\,y\gamma}{{x}^{2}{
\alpha}^{2}}}+{\frac {{\gamma}^{2}x}{{y}^{2}\alpha\,\beta}}-{\frac {{x
}^{2}\gamma}{\beta\,y}}-{\frac {\beta\,{y}^{2}}{\alpha\,x}}+xy
 \right) }
\end{equation}
\subsection{ACSV}

Regular orbit sum.
\[Q(1,1;t)= \Delta -{\frac { \left( -{y}^{2}+x \right)  \left( xy-1 \right)  \left( {x}^{2}-y \right) }{ \left( tx{y}^{2}+t{x}^{2}+ty-1 \right)  \left( -1+x \right)  \left( -1+y \right) \\
\mbox{}xy}}\]

\[Q(u,v;t)= \Delta_{x,y,t} -{\frac { \left( -{y}^{2}+x \right)  \left( xy-1 \right)  \left( {x}^{2}-y \right) }{ \left( tx{y}^{2}+t{x}^{2}+ty-1 \right)  \left( -1+ux \right)  \left( -1+vy \right) \\
\mbox{}xy}}\]


\[Q^{(w)}(1,1;t)= \Delta -{\frac { \left( -{y}^{2}+x \right)  \left( xy-1 \right)  \left( {x}^{2}-y \right) }{ \left({\color{red} ctx{y}^{2}+bt{x}^{2}+aty-1 }\right)  \left( -1+x \right)  \left( -1+y \right) \\
\mbox{}xy}}\]

\begin{verbatim}
H1 := c*t*x*y^2+b*t*x^2+a*t*y-1;
H2 := 1-x;
H3 := 1-y;
\end{verbatim}
\[\displaystyle {\it H1}\, := \,ctx{y}^{2}+bt{x}^{2}+aty-1\]
\[\displaystyle {\it H2}\, := \,1-x\]
\[\displaystyle {\it H3}\, := \,1-y\]

\begin{verbatim}
# On V_\{1,2,3\}
c123 := [1,1,solve(subs(x=1,y=1,H1),t)];
\end{verbatim}
\[\displaystyle {\it c123}\, := \,[1,1, \left( a+b+c \right) ^{-1}]\]


\begin{verbatim} 
with(Groebner):
GB1 := Basis([H1,diff(H1,x)*x-diff(H1,t)*t,diff(H1,y)*y-diff(H1,t)*t],plex(x,y,t));
\end{verbatim}
\[\displaystyle {\it GB1}\, := \,[27\,{a}^{4}b{t}^{3}-{c}^{2},-9\,{a}^{3}b{t}^{2}+{c}^{2}y,-3\,{a}^{2}t+cx]\]

\begin{verbatim} 
solve(%,\{x,y,t\})
\end{verbatim}
\[\displaystyle  \left\{ t={\frac {{\it RootOf} \left( 27\,{{\it \_Z}}^{3}ab-{c}^{2} \right) }{a}},x=3\,{\frac {a{\it RootOf} \left( 27\,{{\it \_Z}}^{3}ab-{c}^{2} \right) }{c}},y\\
\mbox{}=9\,{\frac {ab \left( {\it RootOf} \left( 27\,{{\it
            \_Z}}^{3}ab-{c}^{2} \right)  \right) ^{2}}{{c}^{2}}}
\right\} \]

\begin{verbatim}
# On V1:
c1 := [a^(2/3)/(b^(1/3)*c^(1/3)), a^(1/3)*b^(1/3)/c^(2/3), (1/3)*c^(2/3)/(a^(4/3)*b^(1/3))];
mul(k,k=c1);
subs(x=c1[1],y=c1[2],t=c1[3],GB1);
\end{verbatim}
\[\displaystyle {\it c1}\, := \,[{\frac {{a}^{2/3}}{\sqrt [3]{b}\sqrt [3]{c}}},{\frac {\sqrt [3]{a}\sqrt [3]{b}}{{c}^{2/3}}},1/3\,{\frac {{c}^{2/3}}{{a}^{4/3}\sqrt [3]{b}}}]\]
\[\displaystyle 1/3\,{\frac {1}{\sqrt [3]{a}\sqrt [3]{b}\sqrt [3]{c}}}\]
\[\displaystyle [0,0,0]\]

\begin{verbatim}
# On V12
subs(x=1,H1);
\end{verbatim}
\[\displaystyle ct{y}^{2}+aty+bt-1\]

\begin{verbatim}
GB12 := Basis(subs(x=1,[H1,diff(H1,y)*y-diff(H1,t)*t]),plex(y,t));
\end{verbatim}
\[\displaystyle {\it GB12}\, := \,[ \left( {a}^{2}b-4\,{b}^{2}c
\right) {t}^{2}+4\,bct-c, \left( -{a}^{2}b+4\,{b}^{2}c \right)
t+cya-2\,bc]\]
\begin{verbatim}
c12 := [1,sqrt(b/c),(a*sqrt(b*c)-2*b*c)/(b*(a^2-4*b*c))];
simplify(mul(k,k=c12)) assuming a>0,b>0,c>0;
simplify(subs(x=c12[1],y=c12[2],t=c12[3],GB12)) assuming a>0,b>0,c>0;
\end{verbatim}
\[\displaystyle {\it c12}\, := \,[1, \sqrt{{\frac {b}{c}}},{\frac {a
    \sqrt{bc}-2\,bc}{b \left( {a}^{2}-4\,bc \right)
  }}]\]\[\displaystyle {\frac {a \sqrt{b} \sqrt{c}-2\,bc}{ \sqrt{c}
    \sqrt{b} \left( {a}^{2}-4\,bc \right) }}\]\[\displaystyle [0,0]\]

\begin{verbatim}
# On V13
subs(y=1,H1);
\end{verbatim}

\[\displaystyle bt{x}^{2}+ctx+at-1\]

\begin{verbatim}
GB13 := Basis(subs(y=1,[H1,diff(H1,x)*x-diff(H1,t)*t]),plex(x,t));
\end{verbatim}

\[\displaystyle {\it GB13}\, := \,[ \left( 4\,{a}^{2}b-a{c}^{2} \right) {t}^{2}-4\,abt+b,t \left( 4\,{a}^{2}b-a{c}^{2} \right) +bxc-2\,ab]\]

\begin{verbatim}
c13 := [-sqrt(a)/sqrt(b),1,(c*sqrt(a*b)+2*a*b)/(a*(4*a*b-c^2))];
simplify(mul(k,k=c13)) assuming a>0,b>0,c>0;
simplify(subs(x=c13[1],y=c13[2],t=c13[3],GB13)) assuming a>0,b>0,c>0;
\end{verbatim}

\[\displaystyle {\it c13}\, :=
  \,[-{\frac { \sqrt{a}}{ \sqrt{b}}},1,{\frac {c \sqrt{ab}+2\,ab}{a
      \left( 4\,ab-{c}^{2} \right) }}]\]
\[\displaystyle
  -{\frac {c \sqrt{b} \sqrt{a}+2\,ab}{ \sqrt{a} \sqrt{b} \left(
        4\,ab-{c}^{2} \right) }}\]
\[\displaystyle [0,0]\]
The exponential growth is one of:

\[\displaystyle  \left( a+b+1 \right) ^{n}\]
\[\displaystyle  \left( 3\,\sqrt [3]{ab} \right) ^{n}\]
\[\displaystyle  \left( a+2\, \sqrt{b} \right) ^{n}\]

\begin{figure}\center
\includegraphics[width=.4\textwidth]{growthcurve.jpg}
\caption{$ a+b+1 $ , $3\,\sqrt[3]{ab} $, $a+2\, \sqrt{b}$ for
various values of $a$, $b$}
\end{figure}

\section{Generalizations in higher dimensions}
The combinatorial interpretation of the model (without weights) as a
young tableau of height bounded by 3 is naturally generalized by
considering other heights, say height bounded by~$d+1$. This is in
bijective correspondence with a $d$-dimensional lattice model
restricted to a positive cone, with a step set made of $d+1$ steps:
\[\cT_d=\{e_{i+1}-e_{i}: i=1 \dots d-1\}\cup\{e_1, -e_d\}.\]

These all have D-finite generating functions, from combinatorial
results~\cite{Gessel90}, and expressions as diagonals of rational
functions~\cite{Buetal16}. 

{\it Does the same hold in the weighted case?}

The interpretation as Young tableau in the weighted case is clear, 
but not necessarily motivated. 
\begin{conjecture}
The weighted, $d$-dimensional Tandem model has finite group, 
and a D-finite generating function. 
\end{conjecture}
This is likely easily proved given the appropriate change of
variables as before. 

\mjm{Determine the appropriate change of variables in the higher
  dimension case.}


\section{Conclusion}
\mjm{Different starting points}
{\tiny
\[\displaystyle R\, := \,{\frac {1}{{x}^{2}{y}^{2} \left( at{x}^{2}y+bt{y}^{2}+ctx-xy \right) } \left(  \left( {y}^{-1} \right) ^{i} \left( {x}^{-1} \right) ^{j}xy- \left( {\frac {y}{x}} \right) ^{i}{y}^{3} \left( {x}^{-1} \right) ^{j}- \left( {y}^{-1} \right) ^{i} \left( {\frac {x}{y}} \right) ^{j}{x}^{3}\\
\mbox{}+{x}^{i+4} \left( {\frac {x}{y}} \right) ^{j}y+ \left( {\frac
    {y}{x}} \right) ^{i}{y}^{4+j}x-{x}^{i+3}{y}^{j+3} \right) }\]}
\bibliographystyle{plain}
\bibliography{main.bib}

\section{Appendix}\small
\begin{align*}
Q(1,1;t)=1&+t+2\,{t}^{2}+4\,{t}^{3}+9\,{t}^{4}+21\,{t}^{5}+51\,{t}^{6}+127\,{t}
^{7}+323\,{t}^{8}+835\,{t}^{9}+2188\,{t}^{10}+5798\,{t}^{11}+15511\,{t
}^{12}\\&+41835\,{t}^{13}+113634\,{t}^{14}+310572\,{t}^{15}+853467\,{t}^{
16}+2356779\,{t}^{17}+6536382\,{t}^{18}+18199284\,{t}^{19}\\&+50852019\,{
t}^{20}+142547559\,{t}^{21}+400763223\,{t}^{22}+1129760415\,{t}^{23}+
3192727797\,{t}^{24}\\&+9043402501\,{t}^{25}+25669818476\,{t}^{26}+
73007772802\,{t}^{27}+208023278209\,{t}^{28}+O \left( {t}^{
30} \right) \quad (OEIS A001006)\\[2mm]
Q(x,0;t)=1&+xt+{x}^{2}{t}^{2}+ \left( {x}^{3}+1 \right) {t}^{3}+ \left( {x}^{4}+
3\,x \right) {t}^{4}+ \left( {x}^{5}+6\,{x}^{2} \right) {t}^{5}+
 \left( {x}^{6}+10\,{x}^{3}+5 \right) {t}^{6}+ \left( {x}^{7}+15\,{x}^
{4}+21\,x \right) {t}^{7}\\
&+ \left( {x}^{8}+21\,{x}^{5}+56\,{x}^{2}
 \right) {t}^{8}+ \left( {x}^{9}+28\,{x}^{6}+120\,{x}^{3}+42 \right) {
t}^{9}+ \left( {x}^{10}+36\,{x}^{7}+225\,{x}^{4}+210\,x \right) {t}^{
10}\\&+ \left( {x}^{11}+45\,{x}^{8}+385\,{x}^{5}+660\,{x}^{2} \right) {t}
^{11}+ \left( {x}^{12}+55\,{x}^{9}+616\,{x}^{6}+1650\,{x}^{3}+462
 \right) {t}^{12}+O \left( {t}^{13} \right) \\[2mm]
Q(0,0;t)=1&+{t}^{3}+5\,{t}^{6}+42\,{t}^{9}+462\,{t}^{12}+6006\,{t}^{15}+87516\,
{t}^{18}+1385670\,{t}^{21}+23371634\,{t}^{24}+414315330\,{t}^{27}\\&+
7646001090\,{t}^{30}+O \left( {t}^{
33} \right) 
\end{align*}

\end{document}

